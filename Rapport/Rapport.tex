\documentclass{article}
\usepackage{amsmath}
\usepackage{pgfplots}
\pgfplotsset{compat=1.17}

\begin{document}

\title{Étude du Caractère Pseudo-aléatoire des Décimales de \(e\) par le Test du Chi-deux}
\author{LAMLIH Houssam}
\date{\today}
\maketitle

\section{Introduction}

Le nombre \(e\) est une constante mathématique importante qui représente la base du logarithme naturel. Les décimales de \(e\) ont été étudiées avec intérêt pour leur nature pseudo-aléatoire. Dans ce rapport, nous utilisons le test du chi-deux pour évaluer si les décimales de \(e\) présentent des propriétés pseudo-aléatoires.

\section{Méthodologie}

\subsection{Comptage des chiffres}

Nous initialisons un dictionnaire pour compter les occurrences de chaque chiffre de 0 à 9 dans les décimales de \(e\). En parcourant chaque chiffre de \(e\), nous mettons à jour les comptes dans le dictionnaire.

\subsection{Calcul des fréquences attendues}

Le nombre total de décimales de \(e\) est calculé. Ensuite, les fréquences attendues pour chaque chiffre sont calculées en supposant une répartition uniforme (1/10 pour chaque chiffre).

\subsection{Test du chi-deux}

Le test du chi-deux est utilisé pour évaluer la similarité entre les fréquences observées et les fréquences attendues. La statistique du test du chi-deux est calculée en sommant les carrés des différences normalisées entre les fréquences observées et les fréquences attendues.

\section{Résultats}

\subsection{Statistique du test du chi-deux}

La statistique du test du chi-deux obtenue pour les décimales de \(e\) est de \(X = XXX\), où \(X\) représente la valeur calculée. Cette statistique mesure la différence entre les fréquences observées et les fréquences attendues.

\subsection{Graphique des fréquences observées et attendues}

\begin{center}
\begin{tikzpicture}
% \begin{axis}[
    % ybar,
    % enlargelimits=0.15,
    % ylabel={Fréquence},
    % xlabel={Chiffre},
    % symbolic x coords={0, 1, 2, 3, 4, 5, 6, 7, 8, 9},
    % xtick=data,
    % nodes near coords,
    % nodes near coords align={vertical},
    % ]
% \addplot coordinates {(0, X) (1, X) (2, X) (3, X) (4, X) (5, X) (6, X) (7, X) (8, X) (9, X)};
% \addplot coordinates {(0, X) (1, X) (2, X) (3, X) (4, X) (5, X) (6, X) (7, X) (8, X) (9, X)};
% \legend{Fréquences observées, Fréquences attendues}
% \end{axis}
\end{tikzpicture}
\end{center}

Le graphique ci-dessus montre les fréquences observées des chiffres dans les décimales de \(e\) ainsi que les fréquences attendues (uniformes) pour chaque chiffre.

\section{Discussion}

Le test du chi-deux nous permet d'évaluer la similarité entre les fréquences observées et les fréquences attendues. Une statistique du test du chi-deux plus élevée suggère une plus grande différence entre les fréquences observées et attendues, indiquant un caractère moins pseudo-aléatoire des décimales de \(e\).

Dans notre cas, la statistique du test du chi-deux obtenue est \(X = XXX\). Il serait nécessaire d'évaluer cette valeur par rapport à un seuil de signification approprié pour tirer des conclusions plus définitives sur le caractère pseudo-aléatoire des décimales de \(e\).

\section{Conclusion}

En utilisant le test du chi-deux, nous avons étudié le caractère pseudo-aléatoire des décimales de \(e\). La statistique du test du chi-deux et le graphique des fréquences observées et attendues fournissent des informations sur la similarité entre les fréquences observées et les fréquences attendues.

Cependant, des recherches supplémentaires et une évaluation statistique plus approfondie seraient nécessaires pour tirer des conclusions solides sur le caractère pseudo-aléatoire des décimales de \(e\).

\end{document}
